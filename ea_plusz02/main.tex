% Author: Kovács Bálint-Hunor (Informatika III.)
\documentclass{article}
\usepackage[utf8]{inputenc}
\usepackage[hungarian]{babel}
\usepackage{geometry}
\usepackage{amsmath}
\usepackage{amssymb}
\usepackage{graphicx}
\usepackage{listings} % Kódrészletekhez
\usepackage{xcolor}  % Required for the \color command

% Beállítjuk a nyelvet a kódrészletekhez
\lstdefinestyle{mystyle}{
    language=Matlab,
    basicstyle=\ttfamily,
    keywordstyle=\color{blue},
    commentstyle=\color{green},
    numbers=left,
    numberstyle=\tiny,
    numbersep=5pt,
    frame=single,
    breaklines=true,
    showstringspaces=false,
}

% Beállítjuk a lapméretek
\geometry{top=1in, bottom=1in, left=1in, right=1in}

\title{Mátrix Inverz Kiszámítása Klasszikus és LU Faktorizációval}
\author{Kovács Bálint-Hunor}
\date{2023. október 15.}

\begin{document}
\maketitle

\section{Bevezetés}
Ebben a tanulmányban azt a problémát vizsgáljuk, hogy hogyan lehet kiszámítani
egy $n \times n$ méretű mátrix inverzét. A mátrixot generáljuk az egyszerűség
kedvéért. Az inverz számítást két módszerrel végezzük: a klasszikus mátrix
inverz számítás módszerrel és az LU faktorizációval. Összehasonlítjuk ezeknek a
módszereknek az eredményeit, és elemzést végzünk, hogy melyik módszer
hatékonyabb bizonyos esetekben.

\section{Mátrix Inverziós Módszerek}
A mátrix inverz számítás alapvető művelet a lineáris algebrában. Két közönséges
módszer a mátrix inverz kiszámítására:

\subsection{Klasszikus Mátrix Inverzió}
A klasszikus módszer a mátrix inverz kiszámítására olyan műveleteket használ,
mint a sorcsökkentés, elemi soroperációk és az adjugált mátrix számítása. Ez a
módszer széles körben használt és könnyen megvalósítható.

\subsection{LU Faktorizációs Módszer}
Az LU (Alsó-Felső) faktorizációs módszer a kiindulási mátrixot két háromszög
mátrix szorzataként bontja fel, L (alsó háromszög) és U (felső háromszög). $L
    \times U$ = A. \\ A faktorizáció a Gauss-eliminációs folyamat során történik.
Amint a mátrixot felbontottuk L és U értékekre, könnyebb lineáris
egyenletrendszerek megoldása és az inverz számítása.

Az LU faktorizáció módszer előnyös lehet nagyobb méretű mátrixok esetén.

\section{Mátrix Inverzió Implementációk MATLAB-ban}
\subsection{Klasszikus Mátrix Inverzió Implementáció}

Az alábbi MATLAB kód bemutatja a klasszikus mátrix inverziót.
\begin{lstlisting}[style=mystyle]
    function A_inv = my_traditional_inverse(A)
    [n, ~] = size(A);
    A_inv = eye(n);
    
    for k = 1:n
        % Megkeressuk a pivot elemet
        pivot = A(k, k);
        if abs(pivot) < 1e-10
            error('Matrix is singular.');
        end
        
        % Skalazzuk a sorokat, hogy a pivot elem 1 legyen
        A(k, :) = A(k, :) / pivot;
        A_inv(k, :) = A_inv(k, :) / pivot;
        
        % Nullazzuk a tobbi elemet a k. oszlopban
        for i = 1:n
            if i ~= k
                factor = A(i, k);
                A(i, :) = A(i, :) - factor * A(k, :);
                A_inv(i, :) = A_inv(i, :) - factor * A_inv(k, :);
            end
        end
    end
    end
\end{lstlisting}
A kód magyarázata a következő: \\ A kimeneti mátrixot inicializáljuk az
egységmátrix-al. A ciklusban megkeressük a pivot elemet, és skálázzuk a
sorokat, hogy a pivot elem 1 legyen. Ezután nullázzuk a többi elemet a k.
oszlopban. A ciklus végén megkapjuk az inverz mátrixot.

\subsection{LU Faktorizációs Módszer Implementáció}
Az alábbi MATLAB kód bemutatja az LU faktorizációs módszert.
\begin{lstlisting}[style=mystyle]
    function A_inv = my_lu_inverse(A)
    [n, ~] = size(A);
    L = eye(n);
    U = zeros(n);
    
    for k = 1:n
        U(k, k:n) = A(k, k:n) - L(k, 1:k-1) * U(1:k-1, k:n);
        L(k+1:n, k) = (A(k+1:n, k) - L(k+1:n, 1:k-1) * U(1:k-1, k)) / U(k, k);
    end
    
    I = eye(n);
    Y = L \ I;
    A_inv = U \ Y;
    end
\end{lstlisting}
A kód magyarázata a következő: \\ A kimeneti mátrixot inicializáljuk az
egységmátrix-al. A ciklusban megkeressük a pivot elemet, és skálázzuk a
sorokat, hogy a pivot elem 1 legyen. Ezután nullázzuk a többi elemet a k.
oszlopban. A ciklus végén megkapjuk az inverz mátrixot.

\section{Mátrix Inverzió Tesztelése}
A teszteléshez egy $n \times n$ méretű mátrixot generálunk, ahol $n = 4$.
\begin{lstlisting}[style=mystyle]
clear all; %#ok<*CLALL>
close all;
clc;
N = 4;
A = rand(N, N);
disp('Matrix:');
disp(A);

% Hagyomanyos inverz kiszamitasa
tic;
A_inv_traditional = my_traditional_inverse(A);
time_traditional = toc;

% LU felbontas Crout modszerrel
tic;
A_inv_crout = my_lu_inverse(A);
time_crout = toc;

% Ellenorzes
tolerance = 1e-6;
if norm(A_inv_traditional - A_inv_crout) < tolerance
    disp('Results match within tolerance.');
    disp('Inverse (custom - traditional):');
    disp(A_inv_traditional);
else
    disp('Results do not match.');
    disp('Traditional inverse:');
    disp(A_inv_traditional);
    disp('Inverse with LU decomposition and Crout''s method (custom):');
    disp(A_inv_crout);
end

% Idok kiirasa
fprintf('Time taken for custom traditional matrix inversion: %.6f seconds\n', time_traditional);
fprintf('Time taken for custom LU decomposition with Crout''s method: %.6f seconds\n', time_crout);
\end{lstlisting}
A kimenet a következő:

\begin{lstlisting}[style=mystyle]
Matrix:
0.2725    0.4931    0.1349    0.3069
0.3438    0.9576    0.8482    0.9434
0.6118    0.2305    0.9267    0.6790
0.1700    0.4979    0.7514    0.6095

Results match within tolerance. 
Inverse (custom - traditional): 
3.5065    -2.3519    1.2650    0.4652 
5.7605    -4.9132    -1.6112    6.4988
4.3193    -6.6971    -0.6330    8.8958
-11.0092    12.9265    1.7438    -14.7655

Time taken for custom traditional matrix inversion: 0.004478 seconds
Time taken for custom LU decomposition with Crout's method: 0.030248 seconds
\end{lstlisting}

\section{Összefoglalás}
A tesztelés eredménye alapján látható, hogy mindkét módszer helyes eredményt
adott. A klasszikus módszer gyorsabb volt, mint az LU faktorizációs módszer
ebben az esetben, viszont nagyobb méretű mátrixok esetén az LU faktorizációs
módszer lényegesen hatékonyabb.

Példa egy $1000 \times 1000$ méretű mátrixra (a mátrixot és az inverz mátrixot
ezúttal nem íratom ki, mert túl hosszú lenne):
\begin{lstlisting}[style=mystyle]
Time taken for custom traditional matrix inversion: 16.874812 seconds
Time taken for custom LU decomposition with Crout's method: 0.889298 seconds
\end{lstlisting}
\end{document}
