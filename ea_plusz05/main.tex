\documentclass{article}
\usepackage[utf8]{inputenc}
\usepackage[hungarian]{babel}
\usepackage{amsmath}
\title{MATLAB interpn függvény}
\author{Kovács Bálint-Hunor}
\date{\today}

\begin{document}
\maketitle

\section{\textbf{interpn} függvény magyarázata} Az interpn függvény MATLAB-ban az n dimenziós rácsolt
adatok interpolációját végzi el. A függvény segítségével meg tudjuk becsülni a
függvény értékeit konkrét pontokban, a mintapontokban ismert értékek alapján. A
függvény különböző interpolációs módszereket támogat és lehetővé teszi a
mintapontok tartományán kívüli extrapolációt is.

\subsection{\textbf{interpn} függvény bemeneti paraméterei}
\begin{verbatim}
    Vq = interpn(X1, X2, ..., Xn, V, Xq1, Xq2, ..., Xqn)
    Vq = interpn(V, Xq1, Xq2, ..., Xqn)
    Vq = interpn(V)
    Vq = interpn(V, k)
    Vq = interpn(___, method)
    Vq = interpn(___, method, extrapval)
\end{verbatim}

\subsection{argumentumok magyarázata}
\begin{itemize}
    \item X1, X2, \ldots Xn: mintapontok. Ezek tömbök vagy vektorok, amelyek
          reprezentálják a mintapontok koordinátáit n dimenziós térben. A mintapontok
          egyediek kell legyenek. Ha meg vannak adva, akkor ezeknek meg kell egyezniük a
          V bemenet dimenzióival.
    \item V\@: mintaértékek. Ez egy valós vagy komplex tömb, amely reprezentálja a
          mintapontokban lévő függvényértékeket. A V méreteinek meg kell egyezniük az X1,
          X2, \ldots Xn által definiált rács méreteivel. V több mintaértéket is
          tartalmazhat magasabb dimenziós adatok esetén.
    \item Xq1, Xq2, \ldots, Xqn: lekérdezési pontok. Ezek azok a pontok, amelyekben
          szeretnénk interpolálni a függvényt. Ezek skalárok, vektorok, tömbök vagy
          rácsvektorok lehetnek.
    \item k: finomítási tényező. Megadja, hogy a finomított rács intervallumait hányszor
          kell osztani. A nagyobb k értékek több interpolált pontot eredményeznek a
          mintapontok között. Az alapértelmezett érték 1, ami azt jelenti, hogy nincs
          további finomítás.
    \item method: interpolációs módszer. Megadja, hogy milyen típusú interpolációt kell
          használni. A lehetőségek közé tartozik a `linear' (alapértelmezett),`nearest',
          `pchip', `cubic', `spline' és `makima'. A módszer meghatározza, hogy az
          interpolált értékek hogyan számolódnak ki a mintapontok között.

    \item extrapval: skalár érték, amelyet a mintapontok tartományán kívüli
          lekérdezésekhez rendelünk. Ez az argumentum opcionális. Ha elhagyjuk, akkor az
          interpn visszaad extrapolált értékeket néhány interpolációs módszer esetén és
          NaN értékeket másoknál, a választott módszertől függően.
\end{itemize}

\end{document} % This is the end of the document